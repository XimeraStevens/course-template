\documentclass{ximera}

% add paths to activities with images here
\graphicspath{
{./}
{basic-features/}
}



\newcommand{\RR}{\mathbb R}
\newcommand{\NN}{\mathbb N}
\newcommand{\ZZ}{\mathbb Z}


\DeclareMathOperator{\arcsec}{arcsec}
\DeclareMathOperator{\arccot}{arccot}
\DeclareMathOperator{\arccsc}{arccsc}


\outcome{Familiarization with Desmos graphs}

\title{Interactive graphs with Desmos}

\begin{document}

\begin{abstract}
Abstract goes here.
\end{abstract}

\maketitle



The {\sf graph} command is pretty powerful and rely on \link[Desmos]{https://www.desmos.com/} widget.

The following options are allowed for {\sf graph} command: {\sf panel, xmin, xmax, ymin, ymax, polar, hideXAxis, hideYAxis, xAxisLabel, yAxisLabel, hideXAxisNumbers, hideYAxisNumbers, projectorMode, thinMode, height}.

The simplest example without any options:
\[
\graph{y=x^2}
\]

Advanced example with 3 graphs in polar coordinates (looks bad in a {\sf pdf} document, but great in a browser):
\[
\graph[polar,thinMode,xmin=-40,xmax=40,ymin=-20,ymax=20,hideXAxis]{
r(\theta )=9-3\sin \theta +2\sin \left(3\theta \right)-3\sin (7\theta )+5\cos (2\theta );
r(\theta )=8-2\sin \theta +3\sin \left(3\theta \right)-2\sin (7\theta )+4\cos (2\theta );
r(\theta )=7-1\sin \theta +4\sin \left(3\theta \right)-1\sin (7\theta )+3\cos (2\theta )
}
\]

One more example complex example with derivative and sliding point:
\[
\graph[thinMode,xmin=-5,xmax=5,ymin=-10,ymax=10]{
f(x)=x^3-4x;
g(x)=\frac{d}{dx}f(x);
a=-1;
(a,f(a));
h(x)=g(a)x + (f(a)-g(a)a)
}
\]

Further, look for inspiration here - \link[https://www.desmos.com/math]{https://www.desmos.com/math}.

\end{document}
